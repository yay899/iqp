\documentclass[]{article}

%opening
\title{We kind of need a title}
\author{
  Ternent, James\\
  \texttt{jwternent@wpi.edu}
  \and
  Thompson, Maximilian\\
  \texttt{mthompson2@wpi.edu}
}

\begin{document}
	
	\maketitle
	
	\begin{abstract}
		I'm dying This will contain an abstract at some point, but it’s kind of hard to write one at the moment.
	\end{abstract}
	
	\section{Introduction}
		This section will contain our research questions, and what we did to answer them (research, surveys, data analysis. Prefacing all of this will be the definition of some important terms that are used throughout the paper. Most of our research questions are outlined in another document.
		
	\section{Basics of Artificial Intelligence}
		Artificial Intelligence (AI) in its basic form is a computer program that takes in an input or inputs and produces an output.  An example would be if a program took in how many hours it had been since someone showered and determined whether they were due for a shower or not.  The AI would have a threshold that would trigger a recommendation to shower.  For the purpose of this example we will set it at twenty-four hours.  If it had been sixteen hours since someone showered, it would not tell them to shower, but if it had been twenty-seven hours since they showered, the AI would recommend a shower.  However, we know time is not the only factor in whether someone should shower.  To better improve this theoretical AI, a second input could be added such as whether the person had showered since their last shower.  In general, people should shower after they exercise.  The updated logic in the AI is that it will recommend a shower if the time since the last shower is greater than twenty-four hours or if they have exercised since their last shower. Only one condition has to be true to recommend a shower, but both could be true.  The more inputs or parameters that the AI takes in the better it will get a giving a correct output as long the researcher determined the correct thresholds for the AI.

		Eventually as the AI gets more complex, two problems present themselves.  First, not all problems that an AI tries to solve has a simple linear relation between their inputs and the output.  Second, the researcher may not know the exact relation between the inputs and the output.  This is where a subset of AI called Machine Learning (ML) comes into play.  ML more or less simulates a brain to a far lesser extent.  The way the brain works is it has many neurons that are connected together by synapses.

		% Going to throw some temporary stuff down here.
		\medskip

		Machine learning (ML).

		We really need to work hard to make it clear that we are discussing primarily ML models and not procedural AI. I've basically replaced AI with ML model in almost every part I've been writing below.
		

	\section{General Ethics}
		This section will go into the basic definition of ethics before going into more detail about how ethics apply to medical devices. There will also be some stuff on the legal and ethical considerations of using electronic health records as training data for learning AI.

	\section{Expectations of Medical AI}
		This section will go into how nurses and patients view medical AI, as well as how they expect it to function. This will make heavy use of gathered survey data and highlight both differences between the two groups, and just general areas that could cause some concern if they are not kept in mind while developing medical applications of machine learning.

	\section{Benefits of Medical AI}
		This section will go into what sort of benefits medical AI may be able to provide regardless of what people expect of it. It will bring up how it can lighten the workload of overworked healthcare professionals, and how it may lead to better treatment selection by pulling from data a human would not expect to be relevant.

	\section{Risks of Medical AI}
		Medical ML models carry with them great risk.\textbf{[Citation needed]}

		More often than not ML models are a black box that takes in datasets and outputs values with some prescribed significance. It is difficult for non-ML experts understand the risks and potential failure points of a given ML model, and methods of addressing this lack of interpretability of the model or explainability of the algorithm is still being debated.\cite{10.1145/2858036.2858529, 10.1145/3328519.3329126} This in conjunction with a typical person's initial trust placing little to no emphasis on the actual functionality of a ML model\cite{siau2018building} can make for sudden and unexpected tragedy within medical fields.

		There is immense difficulty of validating the predictions (before acting on them) of a ML model. This creates further risk by obfuscating potential error behind a highly technical veil for non-ML experts. Moreover, unknown error introduced in input datasets is difficult to detect without access to the data itself, as the ML model will continue to output seemingly valid predictions\cite{10.1145/3328519.3329126} Error could be introduced through any number of innocuous vectors (differences in storage systems between health centers, errors digitizing physical records, decay of electronic records over time) or through more malignant ones (such as a targeted attack). The delicacy of handling electronic health records can make it legally difficult to inspect them for these errors\textbf{[Citation needed]}, and the sheer size of the task is prohibitive.

		"Learning to Validate the Predictions of Black Box Machine Learning Models on Unseen Data"\cite{10.1145/3328519.3329126} provides method of detecting dataset shifts without directly examining the dataset itself. \textbf{(I don't entirely understand all the math here, so I'll write this later. Alternatively, James could handle this. --Max)}

	\section{Balance of Regulation Against the Free Market}
		This section will go into detail on the pros and cons of high and low regulation, emphasizing both the massive increase in risk that comes with low regulations and how high regulations could strangle innovation.

\medskip

\bibliographystyle{unsrt}
\bibliography{thompson,ternent}
		
\end{document}

