\documentclass[]{article}

%opening

\begin{document}
	
    \section{Introduction}

    This is the methodology for IRB-20-0598, titled "Survey on the Role of Artificial Intelligence in Healthcare". It is a survey aiming to gauge the general opinion, expectations, and fears regarding medical AI within the nursing student community. We aim to distribute it at nearby nursing schools.

    Bianca Nogrady's \emph{BBC} article \emph{The real risks of artificial intelligence} points out that the risks of artificial intelligence (AI) don't correlate with the fears regarding it. "Rather than worrying about a future AI takeover, the real risk is that we can put too much trust in the smart systems we are building."\cite{bbc2016rroai}. This raises an important question: how aware is the nursing community of the risk of AI? This is one of the questions we seek to answer with our survey. Risks are not limited to difficulty verifying outputs, also including significant increases in financial inequality and increased pollution)\cite{emerj2019roawrtiwwa}, but the difficulty verifying the outputs of AI is what we believe to be most relevant to medical applications.

    We would also like to gauge opinion on more general issues such as the issue of legal culpability following catastrophic failure, and what role machine learning should play within medical proceedings. We believe seeking the opinion of those within the medical field will offer insight into not only how AI is viewed, but also where it should be taken in the future and what restrictions and regulations could lead to safe operation and comfort.
    
    \section{Methodology}

    The survey will be distributed and filled out using Google Forms, as this allows us to reach a wider population and reduces the difficulty and commitment required to complete the survey. Email addresses, IP addresses, and other identifying information will not be collected. However, responses will be individual. Due to the nature of the questions, we do not believe this will create significant risk for the participants. We intend to distribute the survey to nursing students.

    Participants will receive no compensation for completion of the survey, and suffer no consequence for failure to complete or partial completion. No questions are marked as required, and participation may halt at any time with or without submission. We will not be monitoring participants in any way during the process.

    We will use our quantitative questions to sort our data corpus into a number of sets. The free text responses in these sets will be analyzed using thematic analysis as detailed as Braun and Clarke's \emph{Using thematic analysis in psychology}\cite{doi:10.1191/1478088706qp063oa} with an emphasis on latent analysis (although semantic methods will still be employed in some capacity). To compliment this we will also perform make use of grounded theory and axial coding as detailed in Corbin and Strauss' \emph{Grounded theory research: Procedures, canons, and evaluative criteria}\cite{Corbin1990}.

    Additionally, a statistical analysis of the quantitative data will be performed in the hopes of creating further insight in the subject at hand. This will provide a more broad overview of the data corpus that we hope will work with the quantitative analysis of data sets.

    We may chose to directly quote portions of free text responses, however they will not be paired with other responses from that individual and will be anonymized in order to limit potential risk to participants. While we do not anticipate that the resulting data corpus will contain anything that may risk harm to participants, we intend to do our best to eliminate any and all risk.

    While we will work to eliminate bias in the conclusions we draw from the data corpus, we acknowledge that our technical backgrounds may result in some bias that may harm the results of the study. We hope that application of the analysis methods listed above will go a long way to limiting the impact of bias.

    \medskip

    \bibliographystyle{unsrt}
    \bibliography{thompson,ternent}
		
\end{document}

